\documentclass{ctexart}
\usepackage{array}
\usepackage{tabularx}
\usepackage{booktabs}
\usepackage{multirow}
\usepackage{makecell}

\title{Test title}
\author{ Mary\thanks{E-mail:*****@***.com}
\and Ted\thanks{Corresponding author}
\and Louis}
\date{\today}

\begin{document}

\maketitle

A reference to this subsection
\label{sec:this} looks like:
``see section~\ref{sec:this} on
page~\pageref{sec:this}.''

\begin{tabular}{l}
\hline
“天地玄黄,宇宙洪荒。日月盈昃,辰宿列张。”\footnotemark \\
\hline
\end{tabular}
\footnotetext{表格里的名句出自《千字文》。}
效果

\marginpar{\footnotesize 边注较窄,不要写过多文字,最好设置较小的字号。}

\begin{enumerate}
\item An item.
\begin{enumerate}
\item A nested item.\label{itref}
\item[*] A starred item.
\end{enumerate}
\item Reference(\ref{itref}).
\end{enumerate}

\begin{itemize}
    \item An item.
    \begin{itemize}
    \item A nested item.
    \item[+] A `plus' item.
    \item Another item.
    \end{itemize}
    \item Go back to upper level.
\end{itemize}

\begin{description}
    \item[Enumerate] Numbered list.
    \item[Itemize] Non-numbered list.
\end{description}

\renewcommand{\labelitemi}{\ddag}
\renewcommand{\labelitemii}{\dag}
\begin{itemize}
\item First item
\begin{itemize}
\item Subitem
\item Subitem
\end{itemize}
\item Second item
\end{itemize}

\renewcommand{\labelenumi}%
{\Alph{enumi}>}
\begin{enumerate}
\item First item
\item Second item
\end{enumerate}

\begin{center}
Centered text using a
\verb|center| environment.
\end{center}
\begin{flushleft}
Left-aligned text using a
\verb|flushleft| environment.
\end{flushleft}
\begin{flushright}
Right-aligned text using a
\verb|flushright| environment.
\end{flushright}

\centering
Centered text paragraph.
\raggedright
Left-aligned text paragraph.
\raggedleft
Right-aligned text paragraph.

Francis Bacon says:
\begin{quote}
Knowledge is power.
\end{quote}

《木兰诗》:
\begin{quotation}
万里赴戎机,关山度若飞。
朔气传金柝,寒光照铁衣。
将军百战死,壮士十年归。
归来见天子,天子坐明堂。
策勋十二转,赏赐百千强。⋯⋯
\end{quotation}

Rabindranath Tagore's short poem:
\begin{verse}
Beauty is truth's smile
when she beholds her own face in
a perfect mirror.
\end{verse}

\begin{verbatim}
    #include <iostream>
    int main()
    {
    std::cout << "Hello, world!"
    << std::endl;
    return 0;
    }
\end{verbatim}

\begin{verbatim*}
    for (int i=0; i<4; ++i)
    printf("Number %d\n",i);
\end{verbatim*}

\verb|\LaTeX| \\
\verb+(a || b)+ \verb*+(a || b)+

\begin{tabular}{|c|}
center-\\ aligned \\
\end{tabular},
\begin{tabular}[t]{|c|}
top-\\ aligned \\
\end{tabular},
\begin{tabular}[b]{|c|}
bottom-\\ aligned\\
\end{tabular} tabulars.

\begin{tabular}{lcr|p{6em}}
\hline
left & center & right
& par box with fixed width\\
L & C & R & P \\
\hline
\end{tabular}

\begin{tabular}{@{} r@{:}lr @{}}
    \hline
    1 & 1 & one \\
    11 & 3 & eleven \\
    \hline
\end{tabular}

\begin{tabular}{>{\itshape}r<{*}l}
    \hline
    italic & normal \\
    column & column \\
    \hline
\end{tabular}

\newcommand\txt
{a b c d e f g h i}
\begin{tabular}{cp{2em}m{2em}b{2em}}
\hline
pos & \txt & \txt & \txt \\
\hline
\end{tabular}

\begin{tabular*}{14em}%
    {@{\extracolsep{\fill}}|c|c|c|c|}
    \hline
    A & B & C & D \\ \hline
    a & b & c & d \\ \hline
\end{tabular*}

\begin{tabularx}{14em}%
    {|*{4}{>{\centering\arraybackslash}X|}}
    \hline
    A & B & C & D \\ \hline
    a & b & c & d \\ \hline
\end{tabularx}

\begin{tabular}{|c|c|c|}
    \hline
    4 & 9 & 2 \\ \cline{2-3}
    3 & 5 & 7 \\ \cline{1-1}
    8 & 1 & 6 \\ \hline
\end{tabular}

\begin{tabular}{cccc}
    \toprule
    & \multicolumn{3}{c}{Numbers} \\
    \cmidrule{2-4}
    & 1 & 2 & 3 \\
    \midrule
    Alphabet & A & B & C \\
    Roman & I & II& III \\
    \bottomrule
\end{tabular}

\begin{tabular}{|c|c|c|}
    \hline
    1 & 2 & Center \\ \hline
    \multicolumn{2}{|c|}{3} &
    \multicolumn{1}{r|}{Right} \\ \hline
    4 & \multicolumn{2}{c|}{C} \\ \hline
\end{tabular}

\begin{tabular}{ccc}
    \hline
    \multirow{2}{*}{Item} &
    \multicolumn{2}{c}{Value} \\
    \cline{2-3}
    & First & Second \\ \hline
    A & 1 & 2 \\ \hline
\end{tabular}

\begin{tabular}{|c|c|c|}
    \hline
    a & b & c \\ \hline
    a & \multicolumn{1}{@{}c@{}|}
    {\begin{tabular}{c|c}
    e & f \\ \hline
    e & f \\
    \end{tabular}}
    & c \\ \hline
    a & b & c \\ \hline
\end{tabular}

\begin{tabular}{|c|c|}
    \hline
    a & \makecell{d1 \\ d2} \\
    \hline
    b & c \\
    \hline
\end{tabular}

\renewcommand\arraystretch{1.8}
\begin{tabular}{|c|}
\hline
Really loose \\ \hline
tabular rows.\\ \hline
\end{tabular}

\begin{tabular}{c}
    \hline
    Head lines \\[6pt]
    tabular lines \\
    tabular lines \\ \hline
\end{tabular}

|\mbox{Test some words.}|\\
|\makebox[10em]{Test some words.}|\\
|\makebox[10em][l]{Test some words.}|\\
|\makebox[10em][r]{Test some words.}|\\
|\makebox[10em][s]{Test some words.}|

\fbox{Test some words.}\\
\framebox[10em][r]{Test some words.}

\framebox[10em][r]{Test box}\\[1ex]
\setlength{\fboxrule}{1.6pt}
\setlength{\fboxsep}{1em}
\framebox[10em][r]{Test box}

三字经:\parbox[t]{3em}%
{人之初性本善性相近习相远}
\quad
千字文:
\begin{minipage}[b][8ex][t]{4em}
天地玄黄宇宙洪荒
\end{minipage}

\fbox{\begin{minipage}{15em}%
    这是一个垂直盒子的测试。
    \footnote{脚注来自minipage。}
\end{minipage}}

Black \rule{12pt}{4pt} box.
Upper \rule[4pt]{6pt}{8pt} and
lower \rule[-4pt]{6pt}{8pt} box.
A \rule[-.4pt]{3em}{.4pt} line.

\end{document}
