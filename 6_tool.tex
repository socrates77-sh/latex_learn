\documentclass{ctexart}

\usepackage{xcolor}
\usepackage{hyperref}
\usepackage{makeidx}
\makeindex


\begin{document}

\section{Introduction}
Partl~\cite{germenTeX} has proposed that \ldots

Test index.
\index{Test@\textsf{""Test}|(textbf}
\index{Test@\textsf{""Test}!sub@"|sub"||see{Test}}
\newpage
Test index.
\index{Test@\textsf{""Test}|)textbf}

\large\sffamily
{\color[gray]{0.6}
60\% 灰色} \\
{\color[rgb]{0,1,1}
青色}

\large\sffamily
{\color{red} 红色} \\
{\color{blue} 蓝色}

\large\sffamily
{\color{red!40} 40\% 红色}\\
{\color{blue}蓝色
\color{blue!50!black}蓝黑
\color{black}黑色}\\
{\color{-red}红色的互补色}

\sffamily
文字用\textcolor{red}{红色}强调\\
\colorbox[gray]{0.95}{浅灰色背景} \\
\fcolorbox{blue}{yellow}{%
\textcolor{blue}{蓝色边框+文字,%
黄色背景}
}

\url{http://wikipedia.org} \\
\nolinkurl{http://wikipedia.org} \\
\href{http://wikipedia.org}{Wiki}

\begin{thebibliography}{99}
    \bibitem{germenTeX} H.~Partl: \emph{German \TeX},
TUGboat Volume~9, Issue~1 (1988)
\end{thebibliography}

\printindex

\end{document}

\end{document}
