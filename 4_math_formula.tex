\documentclass{article}

\usepackage{amsmath}
\usepackage{amssymb}
\usepackage{amsthm}

\DeclareMathOperator{\argh}{argh}
\DeclareMathOperator*{\nut}{Nut}

\theoremstyle{definition} \newtheorem{law}{Law}
\theoremstyle{plain} \newtheorem{jury}[law]{Jury}
\theoremstyle{remark} \newtheorem*{mar}{Margaret}

\begin{document}

The Pythagorean theorem is
$a^2 + b^2 = c^2$.

The Pythagorean theorem is:
\begin{equation}
a^2 + b^2 = c^2 \label{pythagorean} 
\end{equation}
Equation \eqref{pythagorean} is
called `Gougu theorem' in Chinese.

It's wrong to say
\begin{equation}
1 + 1 = 3 \tag{dumb}
\end{equation}
or
\begin{equation}
1 + 1 = 4 \notag
\end{equation}

\begin{equation*}
a^2 + b^2 = c^2
\end{equation*}
For short:
\[ a^2 + b^2 = c^2 \]
Or if you like the long one:
\begin{displaymath}
a^2 + b^2 = c^2
\end{displaymath}

In text:
$\lim_{n \to \infty}
\sum_{k=1}^n \frac{1}{k^2}
= \frac{\pi^2}{6}$.\\
In display:
\[
\lim_{n \to \infty}
\sum_{k=1}^n \frac{1}{k^2}
= \frac{\pi^2}{6}
\]

$a_1, a_2, \dots, a_n$ \\
$a_1 + a_2 + \cdots + a_n$ \\
$a_1 + a_2 + \dots + a_n$

$p^3_{ij} \qquad
m_\mathrm{Knuth}\qquad
\sum_{k=1}^3 k $\\[5pt]
$a^x+y \neq a^{x+y}\qquad
e^{x^2} \neq {e^x}^2$

$f(x) = x^2 \quad f'(x)
= 2x \quad f''^{2}(x) = 4$ 

In display style:
\[
3/8 \qquad \frac{3}{8}
\qquad \tfrac{3}{8}
\]
In text style:
$1\frac{1}{2}$~hours \qquad
$1\dfrac{1}{2}$~hours

$\sqrt{x} \Leftrightarrow x^{1/2}
\quad \sqrt[3]{2}
\quad \sqrt{x^{2} + \sqrt{y}}$

Pascal's rule is
\[
\binom{n}{k} =\binom{n-1}{k}
+ \binom{n-1}{k-1}
\]

\[
f_n(x) \stackrel{*}{\approx} 1
\]

\[
\lim_{x \rightarrow 0}
\frac{\sin x}{x}=1
\]

$a\bmod b \\
x\equiv a \pmod{b}$

\[\argh 3 = \nut_{x=1} 4x\]

In text:
$\sum_{i=1}^n \quad
\int_0^{\frac{\pi}{2}} \quad
\oint_0^{\frac{\pi}{2}} \quad
\prod_\epsilon $ \\
In display:
\[\sum_{i=1}^n \quad
\int_0^{\frac{\pi}{2}} \quad
\oint_0^{\frac{\pi}{2}} \quad
\prod_\epsilon \]

In text:
$\sum\limits_{i=1}^n \quad
\int\limits_0^{\frac{\pi}{2}} \quad
\prod\limits_\epsilon $ \\
In display:
\[\sum\nolimits_{i=1}^n \quad
\int\limits_0^{\frac{\pi}{2}} \quad
\prod\nolimits_\epsilon \]

\[
\sum_{\substack{0\le i\le n \\
j\in \mathbb{R}}}
P(i,j) = Q(n)
\]
\[
\sum_{\begin{subarray}{l}
0\le i\le n \\
j\in \mathbb{R}
\end{subarray}}
P(i,j) = Q(n)
\]

$\bar{x_0} \quad \bar{x}_0$\\[5pt]
$\vec{x_0} \quad \vec{x}_0$\\[5pt]
$\hat{\mathbf{e}_x} \quad
\hat{\mathbf{e}}_x$

$0.\overline{3} =
\underline{\underline{1/3}}$ \\[5pt]
$\hat{XY} \qquad \widehat{XY}$\\[5pt]
$\vec{AB} \qquad
\overrightarrow{AB}$

$\underbrace{\overbrace{(a+b+c)}^6
\cdot \overbrace{(d+e+f)}^7}
_\text{meaning of life} = 42$

\[ a\xleftarrow{x+y+z} b \]
\[ c\xrightarrow[x<y]{a*b*c}d \]

${a,b,c} \neq \{a,b,c\}$

\[1 + \left(\frac{1}{1-x^{2}}
\right)^3 \qquad
\left.\frac{\partial f}{\partial t}
\right|_{t=0}\]

$\Bigl((x+1)(x-1)\Bigr)^{2}$\\
$\bigl( \Bigl( \biggl( \Biggl( \quad
\bigr\} \Bigr\} \biggr\} \Biggr\} \quad
\big\| \Big\| \bigg\| \Bigg\| \quad
\big\Downarrow \Big\Downarrow
\bigg\Downarrow \Bigg\Downarrow$

\begin{multline}
    a + b + c + d + e + f
    + g + h + i \\
    = j + k + l + m + n\\
    = o + p + q + r + s\\
    = t + u + v + x + z
\end{multline}

\begin{align}
    a & = b + c \\
    & = d + e
\end{align}

\begin{align}
    a ={} & b + c \\
    ={} & d + e + f + g + h + i
    + j + k + l \notag \\
    & + m + n + o \\
    ={} & p + q + r + s
\end{align}

\begin{align}
    a &=1 & b &=2 & c &=3 \\
    d &=-1 & e &=-2 & f &=-5
\end{align}

\begin{gather}
    a = b + c \\
    d = e + f + g \\
    h + i = j + k \notag \\
    l + m = n
\end{gather}

\begin{equation}
    \begin{aligned}
    a &= b + c \\
    d &= e + f + g \\
    h + i &= j + k \\
    l + m &= n
    \end{aligned}
\end{equation}

\[ \mathbf{X} = \left(
\begin{array}{cccc}
x_{11} & x_{12} & \ldots & x_{1n}\\
x_{21} & x_{22} & \ldots & x_{2n}\\
\vdots & \vdots & \ddots & \vdots\\
x_{n1} & x_{n2} & \ldots & x_{nn}\\
\end{array} \right) \]

\[ |x| = \left\{
\begin{array}{rl}
-x & \text{if } x < 0,\\
0 & \text{if } x = 0,\\
x & \text{if } x > 0.
\end{array} \right. \]

\[ |x| =
\begin{cases}
-x & \text{if } x < 0,\\
0 & \text{if } x = 0,\\
x & \text{if } x > 0.
\end{cases} \]

\[
\begin{matrix}
1 & 2 \\ 3 & 4
\end{matrix} \qquad
\begin{bmatrix}
x_{11} & x_{12} & \ldots & x_{1n}\\
x_{21} & x_{22} & \ldots & x_{2n}\\
\vdots & \vdots & \ddots & \vdots\\
x_{n1} & x_{n2} & \ldots & x_{nn}\\
\end{bmatrix}
\]

\[
\mathbf{H}=
\begin{bmatrix}
\dfrac{\partial^2 f}{\partial x^2} &
\dfrac{\partial^2 f}
{\partial x \partial y} \\[8pt]
\dfrac{\partial^2 f}
{\partial x \partial y} &
\dfrac{\partial^2 f}{\partial y^2}
\end{bmatrix}
\]

\[
\int_a^b f(x)\mathrm{d}x
\qquad
\int_a^b f(x)\,\mathrm{d}x
\]

\newcommand\diff{\,\mathrm{d}}
\begin{gather*}
\int\int f(x)g(y)
\diff x \diff y \\
\int\!\!\!\int
f(x)g(y) \diff x \diff y \\
\iint f(x)g(y) \diff x \diff y \\
\iint\quad \iiint\quad \idotsint
\end{gather*}

$\mathcal{R} \quad \mathfrak{R}
\quad \mathbb{R}$
\[\mathcal{L}
= -\frac{1}{4}F_{\mu\nu}F^{\mu\nu}\]
$\mathfrak{su}(2)$ and
$\mathfrak{so}(3)$ Lie algebra

\[
r = \frac
{\sum_{i=1}^n (x_i- x)(y_i- y)}
{\displaystyle \left[
\sum_{i=1}^n (x_i-x)^2
\sum_{i=1}^n (y_i-y)^2
\right]^{1/2} }
\]

$\mu, M \qquad
\mathbf{\mu}, \mathbf{M}$
\qquad {\boldmath$\mu, M$}

$\mu, M \qquad
\boldsymbol{\mu}, \boldsymbol{M}$

\newtheorem{mythm}{My Theorem}[section]
\begin{mythm}\label{thm:light}
The light speed in vacuum
is $299,792,458\,\mathrm{m/s}$.
\end{mythm}
\begin{mythm}[Energy-momentum relation]
The relationship of energy,
momentum and mass is
\[E^2 = m_0^2 c^4 + p^2 c^2\]
where $c$ is the light speed
described in theorem \ref{thm:light}.
\end{mythm}

\begin{law}\label{law:box}
Don't hide in the witness box.
\end{law}
\begin{jury}[The Twelve]
It could be you! So beware and
see law~\ref{law:box}.\end{jury}
\begin{jury}
You will disregard the last
statement.\end{jury}
\begin{mar}No, No, No\end{mar}
\begin{mar}Denis!\end{mar}

\begin{proof}
For simplicity, we use
\[
E=mc^2
\]
That's it.
\end{proof}

\begin{proof}
For simplicity, we use
\[
E=mc^2 \qedhere
\]
\end{proof}

\begin{proof}
Assuming $\gamma
= 1/\sqrt{1-v^2/c^2}$, then
\begin{align*}
E &= \gamma m_0 c^2 \\
p &= \gamma m_0v \qedhere
\end{align*}
\end{proof}

\begin{proof}
For simplicity, we use
\begin{equation}
E=mc^2.
\end{equation}
\end{proof}

\renewcommand{\qedsymbol}%
{\rule{1ex}{1.5ex}}
\begin{proof}
For simplicity, we use
\[
E=mc^2 \qedhere
\]
\end{proof}

\end{document}
